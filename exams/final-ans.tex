\documentclass[10pt]{article}

\usepackage{mathtools, amssymb, bm}
\usepackage{microtype}
\usepackage[utf8]{inputenc}
\usepackage[margin = 0.75in]{geometry}
\usepackage{booktabs}
\usepackage{graphicx}
\usepackage{xcolor}
\usepackage{tikzsymbols}
\usepackage[hidelinks]{hyperref}
\usepackage{titlesec}



\titleformat{\section}{\normalsize\bfseries}{\thesection}{1em}{}
% \titleformat{\section}{\large\bfseries}{\thesection}{1em}{}
\setcounter{secnumdepth}{0}
% \numberwithin{equation}{section}

\definecolor{colabcol}{HTML}{960018}
\newcommand{\mycolab}[1]{\textcolor{colabcol}{\textsl{Collaborators:}} #1 \\ }
\newcommand{\mycolaba}[1]{\textcolor{colabcol}{\textsl{Collaborators:}} #1}

\title{
    {\Large Final Exam}
}
\author{
    {\normalsize Aiden Kenny}\\
    {\normalsize STAT GR5205: Linear Regression Models}\\
    {\normalsize Columbia University}
}
\date{\normalsize December 21, 2020}

\begin{document}

\maketitle

% \newcommand{\rx}{\mathcal{X}}
\newcommand{\rx}{X}
% \newcommand{\ry}{\mathcal{Y}}
\newcommand{\ry}{Y}
% \newcommand{\rz}{\mathcal{Z}}
\newcommand{\rz}{Z}
\newcommand{\E}{\mathbb{E}}
% \newcommand{\E}{\mathrm{E}}

%' ============================================================================================================================================================
\section{Question 1} \noindent
Let \(\ry\) denote per-capita gross metropolitan product (GMP), in dollars per person per year, and \(\rx\) denote population, in people. 
The realized values of these random variables are respectively given by the \(n\)-vectors \(\mathbf{y}\) and \(\mathbf{x}\), where \(n = 366\).
\begin{enumerate}
    \item The predictor variable is given by \(\rz \coloneqq \log_{10}\rx\), and the response is \(\ry\). We can see that the population is being transformed 
    by taking the logarithm (with base 10). 
    \item Our estimated model is given by 
    \begin{align}
        % \hat{\ry}
        \E(\ry \,|\, x)
        % \mathbb{E}[\ry \,|\, \rx]
        % = -23306 + 10246 \rz
        = -23306 + 10246 \log_{10} x.
    \end{align}
    \item We have \(\E(\ry \,|\, 1,000,000) = 38170\) and \(\E(\ry \,|\, 200,000) = 31008.35\).
    These answers make sense, a city with a larger population will have a higher GMP per-capita. \\[0.5em]
    \texttt{-23306 + 10246 * log(c(1000000, 200000), 10)}
    \item We cannot give an estimate of \(\E(\ry \,|\, 0)\) because \(\log_{10} 0\) is undefined.
    \item A \(95\%\) confidence interval for \(\beta_1\), denoted as \(\mathcal{I}_{\beta_1}\), is
    \begin{align}
        \mathcal{I}_{\beta_1}
        = \left( \hat{\beta}_1 - t \cdot \mathrm{se}(\hat{\beta}_1), \hat{\beta}_1 + t \cdot \mathrm{se}(\hat{\beta}_1) \right)
        = \big( 10246 - 1.967 \cdot 900, 10246 + 1.967 \cdot 900 \big)
        = \big( 8475.7, 12016.3 \big).
    \end{align}
    The values \(\hat{\beta}_1 = 10246\) and \(\mathrm{se}(\hat{\beta}_1) = 900\) can be found in the \texttt{R} output, and the value 
    \(t = T_{364}^{-1}(0.975) = 1.967\) can be found using the \texttt{qt()} function in \texttt{R}. \\[0.5em]
    \texttt{qt(0.975, 364)}\\[0.5em]
    \texttt{10246 + 1.967 * 900 * c(-1, 1)}
    \item From the \texttt{\#{}\#{}Residual standard error} section, we have \(\hat{\sigma}^2 = (7930/364)^2 = 474.6173\). \\[0.5em]
    % \texttt{7930\textasciicircum{}2}
    \texttt{(7930 / 364)\textasciicircum{}2}
    \item You cannot find the sample variance of \(\rx\) from the \texttt{R} output. We are never considering the value of 
    \(\mathrm{Var}(\rz)\) when constructing the model because we are never treating \(\rz\) as a random variable. We instead are treating it as 
    a set of fixed values \(\mathbf{z}\), either observed before or after the model's design is chosen. When we estimate \(\sigma^2\) in 
    the linear model, we are estimating \(\mathrm{Var}(\epsilon)\), the residuals of the model. And since we cannot make any inferences about 
    \(\mathrm{Var}(\rz)\), we cannot make any inferences about \(\mathrm{Var}(\rx)\) either. 
    % ; practically \(\mathbf{z}\) is either the observed values of the transformed predictor variable before fitting the model, 
    % or the conditional values of the random variable \(\ry \,|\, \mathbf{z}\). 
    \item There are multiple components of the \texttt{R} output that test the hypothesis \(H_0 : \beta_1 = 0\) against \(H_A : \beta_1 \neq 0\). 
    Remember, the output is testing the hypothesis that \(\ry\) and \(\rz\) have a linear relationship, \textsl{not} \(\ry\) and \(\rx\). 
    There are two tests that \texttt{R} runs when using the \texttt{lm()} function: the \(t\) test and the ANOVA test. The \(p\)-palue for the \(t\) test
    is found in the right-most column, \texttt{Pr(>|t|)}, of the \texttt{\#{}\#{}Coefficients} section, and is given by \texttt{<2e-16}. 
    The \(p\)-value for the ANVOA test is found in the last entry in the output, in the \texttt{\#{}\#{}F-statistic} section, and is also given by \texttt{<2e-16}
    (\texttt{R} will estimate the value if it is too small). In both cases, we reject \(H_0\), and it seems that there is indeed a linear relationship 
    between \(\ry\) and \(\rz\) (\(=\log_{10}\rx\)). 
\end{enumerate}

\end{document}